% Options for packages loaded elsewhere
\PassOptionsToPackage{unicode}{hyperref}
\PassOptionsToPackage{hyphens}{url}
%
\documentclass[
  12pt,
]{article}
\usepackage{lmodern}
\usepackage{amssymb,amsmath}
\usepackage{ifxetex,ifluatex}
\ifnum 0\ifxetex 1\fi\ifluatex 1\fi=0 % if pdftex
  \usepackage[T1]{fontenc}
  \usepackage[utf8]{inputenc}
  \usepackage{textcomp} % provide euro and other symbols
\else % if luatex or xetex
  \usepackage{unicode-math}
  \defaultfontfeatures{Scale=MatchLowercase}
  \defaultfontfeatures[\rmfamily]{Ligatures=TeX,Scale=1}
  \setmainfont[]{Times New Roman}
\fi
% Use upquote if available, for straight quotes in verbatim environments
\IfFileExists{upquote.sty}{\usepackage{upquote}}{}
\IfFileExists{microtype.sty}{% use microtype if available
  \usepackage[]{microtype}
  \UseMicrotypeSet[protrusion]{basicmath} % disable protrusion for tt fonts
}{}
\makeatletter
\@ifundefined{KOMAClassName}{% if non-KOMA class
  \IfFileExists{parskip.sty}{%
    \usepackage{parskip}
  }{% else
    \setlength{\parindent}{0pt}
    \setlength{\parskip}{6pt plus 2pt minus 1pt}}
}{% if KOMA class
  \KOMAoptions{parskip=half}}
\makeatother
\usepackage{xcolor}
\IfFileExists{xurl.sty}{\usepackage{xurl}}{} % add URL line breaks if available
\IfFileExists{bookmark.sty}{\usepackage{bookmark}}{\usepackage{hyperref}}
\hypersetup{
  pdftitle={Effects of a Defaunation Gradient on Tropical Forest Structure in Ivindo National Park, Gabon},
  pdfauthor={Tashsa Griffiths, Israel Golden, Aubrey Knier, and Mishka Malinowski},
  hidelinks,
  pdfcreator={LaTeX via pandoc}}
\urlstyle{same} % disable monospaced font for URLs
\usepackage[margin=2.54cm]{geometry}
\usepackage{longtable,booktabs}
% Correct order of tables after \paragraph or \subparagraph
\usepackage{etoolbox}
\makeatletter
\patchcmd\longtable{\par}{\if@noskipsec\mbox{}\fi\par}{}{}
\makeatother
% Allow footnotes in longtable head/foot
\IfFileExists{footnotehyper.sty}{\usepackage{footnotehyper}}{\usepackage{footnote}}
\makesavenoteenv{longtable}
\usepackage{graphicx,grffile}
\makeatletter
\def\maxwidth{\ifdim\Gin@nat@width>\linewidth\linewidth\else\Gin@nat@width\fi}
\def\maxheight{\ifdim\Gin@nat@height>\textheight\textheight\else\Gin@nat@height\fi}
\makeatother
% Scale images if necessary, so that they will not overflow the page
% margins by default, and it is still possible to overwrite the defaults
% using explicit options in \includegraphics[width, height, ...]{}
\setkeys{Gin}{width=\maxwidth,height=\maxheight,keepaspectratio}
% Set default figure placement to htbp
\makeatletter
\def\fps@figure{htbp}
\makeatother
\setlength{\emergencystretch}{3em} % prevent overfull lines
\providecommand{\tightlist}{%
  \setlength{\itemsep}{0pt}\setlength{\parskip}{0pt}}
\setcounter{secnumdepth}{5}

\title{Effects of a Defaunation Gradient on Tropical Forest Structure in Ivindo
National Park, Gabon}
\usepackage{etoolbox}
\makeatletter
\providecommand{\subtitle}[1]{% add subtitle to \maketitle
  \apptocmd{\@title}{\par {\large #1 \par}}{}{}
}
\makeatother
\subtitle{\url{https://github.com/israelgolden/GoldenGriffithsKnierMalinowski_ENV872_EDA_FinalProject}}
\author{Tashsa Griffiths, Israel Golden, Aubrey Knier, and Mishka Malinowski}
\date{}

\begin{document}
\maketitle

\newpage
\tableofcontents 
\newpage
\listoftables 
\newpage
\listoffigures 
\newpage

\hypertarget{rationale-and-research-questions}{%
\section{\texorpdfstring{\textbf{Rationale and Research
Questions}}{Rationale and Research Questions}}\label{rationale-and-research-questions}}

Tropical forests throughout the world are experiencing changes in forest
structure and ecosystem services due to increasing hunting pressure,
resulting in plummeting animal populations\textsuperscript{1}. This
phenomenon known as defaunation has cascading effects throughout
ecosystems due to the disruption of intricate plant-animal interactions
that are responsible for shaping tropical forests\textsuperscript{1}.
Plant-animal interactions such as seed dispersal, seed predation,
trampling, herbivory, and nutrient translocation are necessary to shape
forests. Through positive interactions such as the distribution of seeds
and nutrients and antagonistic interactions such as herbivory and
trampling - resulting in the opening up of the understory, plant-animal
interactions create opportunities for a variety of species to succeed in
the forest and increase plant diversity, richness, and ecosystem
services\textsuperscript{2}. For example, in the Afrotropical forests of
LuiKotale in the Congo Basin, 95\% of the trees in this area depend on
animals for dispersal demonstrating the necessity of plant-animal
interactions in these systems\textsuperscript{3}. The alteration or loss
of faunal communities in tropical forests has led to ``Empty Forest
Syndrome'', where a forest appears to be intact, but the animal
community is so depleated or non-existent that the forest no longer
functions as it did resulting in changes in ecosystem services, such as
carbon storage\textsuperscript{4}.

As defaunation continues to increase globally, there is little
understanding of the long-term effects of defaunation on tropical forest
diversity, ecosystem services and specifically carbon storage
capabilities. Tropcial forests are responsible for sequestering
\textasciitilde40\% of the world's aboveground
carbon\textsuperscript{4}. Therefore, the effects of defaunation on
tropical forests may have detrimental effects for global carbon storage
and climate change projections. Further research is necessary to
understand the intricate interactions between defaunation, tropical
forests, and ecosystem services to illuminate these relationships and
advocate for policy changes and resource management. However, it is
essential to highlight that the underlying causes of defaunation are
top-down driven by the global economy, government regulations and
incentives, access to income and livelihoods,and ultimately quality of
life.

\hypertarget{research-questions}{%
\subsection{\texorpdfstring{\textbf{Research
Questions}}{Research Questions}}\label{research-questions}}

\textbf{Overarching Research Question}

How does defaunation affect forest structure and composition in Ivindo
National Park, Gabon?

\emph{RQ1 \& RQ2 - Forest Structure} Does average diameter at breast
height (DBH) change along a defaunation gradient?

Are there differences in Basal Area across a defaunation gradient?

\emph{RQ3 - Species Composition} Does species composition change along a
defaunation gradient?

\hypertarget{study-overview-site}{%
\section{Study Overview \& Site}\label{study-overview-site}}

Gabon located in central western Africa provides an ideal study site for
these research questions as the second most forested country in the
world (Figure 1/Map 1?). Afrotropical forests extend throughout Gabon,
known as one of the last strongholds in this ecosystem for several
endemic species including the forest elephant \emph{(Loxodonta
cyclotis)}. Ivindo National Park, one of 14 national parks and
presidential reserves in Gabon lies on the outskirts of several villages
providing an excellent location to understand the interactions of
hunting pressure within tropical forests (Figure 2/Map 2?). A previous
study by Koerner et al.~in 2016 in this area demonstrated the existence
of a defaunation graident radiating away from the villages and into
Ivindo national park. The results of the study showed that distance from
villages could be used as a proxy for defuantion and that every 10
kilometers traveled away from villages mammal richness would increase by
1.5 species\textsuperscript{5}. Therefore, in 2020 a project was started
by the Poulsen Ecology Lab to establish forest plots along the
defaunation gradient to further explore the relationship between forest
structure and defaunation. Up to date, 10 sites with a paired-plot
design have been established (plots, n = 20). Within the twenty 50 x 50m
plots all trees above 1.5 meters in height have been tagged and
measured, in six of these plots the trees have also been identified. Our
analyses will focus on the complete data from this subset of 6
plots(Figure 3/Map 3?). Makokou, the largest town in this area also
considered a regional capital, is indicated on the map to demonstrate
hunting pressure and indicate that the most defaunated plots are those
closest to Makokou while intact forests are farthest from Makokou.

\begin{figure}
\centering
\includegraphics{GoldenGriffithsKnierMalinowski_ENV872_Project_files/figure-latex/Map of Gabon-1.pdf}
\caption{Map of Africa with Gabon Indicated}
\end{figure}

\begin{figure}
\centering
\includegraphics{GoldenGriffithsKnierMalinowski_ENV872_Project_files/figure-latex/Detailed Map of Gabon with parks-1.pdf}
\caption{Gabon's National Parks}
\end{figure}

\begin{figure}
\centering
\includegraphics{GoldenGriffithsKnierMalinowski_ENV872_Project_files/figure-latex/Map of Ivindo with plots-1.pdf}
\caption{Forest Plots along a Defaunation Gradient in Ivindo National
Park, Gabon}
\end{figure}

\newpage

\hypertarget{dataset-information}{%
\section{Dataset Information}\label{dataset-information}}

This dataset is provided by the Pouslen Tropical Ecology Lab here at
Duke. The dimensions and variable information of the raw dataset are
below:

\begin{verbatim}
## [1] 45681    21
\end{verbatim}

\begin{longtable}[]{@{}llll@{}}
\caption{Raw Dataset Information}\tabularnewline
\toprule
\begin{minipage}[b]{0.13\columnwidth}\raggedright
Variable\strut
\end{minipage} & \begin{minipage}[b]{0.37\columnwidth}\raggedright
Description\strut
\end{minipage} & \begin{minipage}[b]{0.19\columnwidth}\raggedright
Unit\strut
\end{minipage} & \begin{minipage}[b]{0.20\columnwidth}\raggedright
Range\strut
\end{minipage}\tabularnewline
\midrule
\endfirsthead
\toprule
\begin{minipage}[b]{0.13\columnwidth}\raggedright
Variable\strut
\end{minipage} & \begin{minipage}[b]{0.37\columnwidth}\raggedright
Description\strut
\end{minipage} & \begin{minipage}[b]{0.19\columnwidth}\raggedright
Unit\strut
\end{minipage} & \begin{minipage}[b]{0.20\columnwidth}\raggedright
Range\strut
\end{minipage}\tabularnewline
\midrule
\endhead
\begin{minipage}[t]{0.13\columnwidth}\raggedright
E\strut
\end{minipage} & \begin{minipage}[t]{0.37\columnwidth}\raggedright
Field expedition season\strut
\end{minipage} & \begin{minipage}[t]{0.19\columnwidth}\raggedright
Season-Year\strut
\end{minipage} & \begin{minipage}[t]{0.20\columnwidth}\raggedright
Winter - Summer 2021\strut
\end{minipage}\tabularnewline
\begin{minipage}[t]{0.13\columnwidth}\raggedright
Data\_entry\strut
\end{minipage} & \begin{minipage}[t]{0.37\columnwidth}\raggedright
Name of individual inputting data to Excel\strut
\end{minipage} & \begin{minipage}[t]{0.19\columnwidth}\raggedright
Name\strut
\end{minipage} & \begin{minipage}[t]{0.20\columnwidth}\raggedright
\strut
\end{minipage}\tabularnewline
\begin{minipage}[t]{0.13\columnwidth}\raggedright
Date..dd.mm.yyyy.\strut
\end{minipage} & \begin{minipage}[t]{0.37\columnwidth}\raggedright
Date of Excel data entry\strut
\end{minipage} & \begin{minipage}[t]{0.19\columnwidth}\raggedright
Date/Month/Year\strut
\end{minipage} & \begin{minipage}[t]{0.20\columnwidth}\raggedright
March 2021 - November 2022\strut
\end{minipage}\tabularnewline
\begin{minipage}[t]{0.13\columnwidth}\raggedright
File\_name\strut
\end{minipage} & \begin{minipage}[t]{0.37\columnwidth}\raggedright
Photo file name of field data sheet\strut
\end{minipage} & \begin{minipage}[t]{0.19\columnwidth}\raggedright
.JPG\strut
\end{minipage} & \begin{minipage}[t]{0.20\columnwidth}\raggedright
\strut
\end{minipage}\tabularnewline
\begin{minipage}[t]{0.13\columnwidth}\raggedright
Date..dd.mm.yyyy..1\strut
\end{minipage} & \begin{minipage}[t]{0.37\columnwidth}\raggedright
Data of field data collection\strut
\end{minipage} & \begin{minipage}[t]{0.19\columnwidth}\raggedright
Data of field data collection\strut
\end{minipage} & \begin{minipage}[t]{0.20\columnwidth}\raggedright
June - January 2021\strut
\end{minipage}\tabularnewline
\begin{minipage}[t]{0.13\columnwidth}\raggedright
Note\_taker\strut
\end{minipage} & \begin{minipage}[t]{0.37\columnwidth}\raggedright
Name of individual recording field data\strut
\end{minipage} & \begin{minipage}[t]{0.19\columnwidth}\raggedright
Name\strut
\end{minipage} & \begin{minipage}[t]{0.20\columnwidth}\raggedright
\strut
\end{minipage}\tabularnewline
\begin{minipage}[t]{0.13\columnwidth}\raggedright
Project\strut
\end{minipage} & \begin{minipage}[t]{0.37\columnwidth}\raggedright
Defaunated forest (DF) or intact forest (IF) plot\strut
\end{minipage} & \begin{minipage}[t]{0.19\columnwidth}\raggedright
Category\strut
\end{minipage} & \begin{minipage}[t]{0.20\columnwidth}\raggedright
DF or IF\strut
\end{minipage}\tabularnewline
\begin{minipage}[t]{0.13\columnwidth}\raggedright
Plot\strut
\end{minipage} & \begin{minipage}[t]{0.37\columnwidth}\raggedright
Unique plot identification\strut
\end{minipage} & \begin{minipage}[t]{0.19\columnwidth}\raggedright
Category\strut
\end{minipage} & \begin{minipage}[t]{0.20\columnwidth}\raggedright
1-6 A-B\strut
\end{minipage}\tabularnewline
\begin{minipage}[t]{0.13\columnwidth}\raggedright
Grid\strut
\end{minipage} & \begin{minipage}[t]{0.37\columnwidth}\raggedright
Within-plot grid where data were collected\strut
\end{minipage} & \begin{minipage}[t]{0.19\columnwidth}\raggedright
Category\strut
\end{minipage} & \begin{minipage}[t]{0.20\columnwidth}\raggedright
\strut
\end{minipage}\tabularnewline
\begin{minipage}[t]{0.13\columnwidth}\raggedright
TAG\_SUM\strut
\end{minipage} & \begin{minipage}[t]{0.37\columnwidth}\raggedright
The most unique identifier, using plot grid and plant tag\strut
\end{minipage} & \begin{minipage}[t]{0.19\columnwidth}\raggedright
Category\strut
\end{minipage} & \begin{minipage}[t]{0.20\columnwidth}\raggedright
\strut
\end{minipage}\tabularnewline
\begin{minipage}[t]{0.13\columnwidth}\raggedright
Plant\_tag\strut
\end{minipage} & \begin{minipage}[t]{0.37\columnwidth}\raggedright
Identifer assigned to each sample\strut
\end{minipage} & \begin{minipage}[t]{0.19\columnwidth}\raggedright
Letter-Number Combination\strut
\end{minipage} & \begin{minipage}[t]{0.20\columnwidth}\raggedright
\strut
\end{minipage}\tabularnewline
\begin{minipage}[t]{0.13\columnwidth}\raggedright
X\_coord\strut
\end{minipage} & \begin{minipage}[t]{0.37\columnwidth}\raggedright
X coordinate of sample location\strut
\end{minipage} & \begin{minipage}[t]{0.19\columnwidth}\raggedright
Degrees\strut
\end{minipage} & \begin{minipage}[t]{0.20\columnwidth}\raggedright
0.00 - 9.80\strut
\end{minipage}\tabularnewline
\begin{minipage}[t]{0.13\columnwidth}\raggedright
Y\_coord\strut
\end{minipage} & \begin{minipage}[t]{0.37\columnwidth}\raggedright
Y coordinate of sample location\strut
\end{minipage} & \begin{minipage}[t]{0.19\columnwidth}\raggedright
Degrees\strut
\end{minipage} & \begin{minipage}[t]{0.20\columnwidth}\raggedright
0.00 - 8.75\strut
\end{minipage}\tabularnewline
\begin{minipage}[t]{0.13\columnwidth}\raggedright
Tool\strut
\end{minipage} & \begin{minipage}[t]{0.37\columnwidth}\raggedright
Tool used to measure diameter\strut
\end{minipage} & \begin{minipage}[t]{0.19\columnwidth}\raggedright
Category\strut
\end{minipage} & \begin{minipage}[t]{0.20\columnwidth}\raggedright
DBH or CP\strut
\end{minipage}\tabularnewline
\begin{minipage}[t]{0.13\columnwidth}\raggedright
POM\strut
\end{minipage} & \begin{minipage}[t]{0.37\columnwidth}\raggedright
Point of measurement for diameter\strut
\end{minipage} & \begin{minipage}[t]{0.19\columnwidth}\raggedright
Meters\strut
\end{minipage} & \begin{minipage}[t]{0.20\columnwidth}\raggedright
0.00 - 11.00\strut
\end{minipage}\tabularnewline
\begin{minipage}[t]{0.13\columnwidth}\raggedright
DBH.mm\strut
\end{minipage} & \begin{minipage}[t]{0.37\columnwidth}\raggedright
Diameter at breast height (DBH)\strut
\end{minipage} & \begin{minipage}[t]{0.19\columnwidth}\raggedright
Millimeters\strut
\end{minipage} & \begin{minipage}[t]{0.20\columnwidth}\raggedright
0.00 - 173.00\strut
\end{minipage}\tabularnewline
\begin{minipage}[t]{0.13\columnwidth}\raggedright
Height..meters.\strut
\end{minipage} & \begin{minipage}[t]{0.37\columnwidth}\raggedright
Height of plant\strut
\end{minipage} & \begin{minipage}[t]{0.19\columnwidth}\raggedright
Meters\strut
\end{minipage} & \begin{minipage}[t]{0.20\columnwidth}\raggedright
0.07 - 70.00\strut
\end{minipage}\tabularnewline
\begin{minipage}[t]{0.13\columnwidth}\raggedright
Type\_Field\strut
\end{minipage} & \begin{minipage}[t]{0.37\columnwidth}\raggedright
Vegetation type or size class of plant\strut
\end{minipage} & \begin{minipage}[t]{0.19\columnwidth}\raggedright
Category\strut
\end{minipage} & \begin{minipage}[t]{0.20\columnwidth}\raggedright
Seedling, Sapling, Liana, Tree\strut
\end{minipage}\tabularnewline
\begin{minipage}[t]{0.13\columnwidth}\raggedright
Note\_Field\strut
\end{minipage} & \begin{minipage}[t]{0.37\columnwidth}\raggedright
Miscellaneous field notes\strut
\end{minipage} & \begin{minipage}[t]{0.19\columnwidth}\raggedright
Phrase\strut
\end{minipage} & \begin{minipage}[t]{0.20\columnwidth}\raggedright
\strut
\end{minipage}\tabularnewline
\begin{minipage}[t]{0.13\columnwidth}\raggedright
ID\strut
\end{minipage} & \begin{minipage}[t]{0.37\columnwidth}\raggedright
Latin species identification\strut
\end{minipage} & \begin{minipage}[t]{0.19\columnwidth}\raggedright
Name\strut
\end{minipage} & \begin{minipage}[t]{0.20\columnwidth}\raggedright
\strut
\end{minipage}\tabularnewline
\begin{minipage}[t]{0.13\columnwidth}\raggedright
Treatment\strut
\end{minipage} & \begin{minipage}[t]{0.37\columnwidth}\raggedright
Future plot treatments (fungicide/insecticide)\strut
\end{minipage} & \begin{minipage}[t]{0.19\columnwidth}\raggedright
Category\strut
\end{minipage} & \begin{minipage}[t]{0.20\columnwidth}\raggedright
LMC, LME, MME, MMC\strut
\end{minipage}\tabularnewline
\bottomrule
\end{longtable}

With such a large dataset, data cleaning and wrangling was an essential
process for creating a manageable dataset that was relevant for
answering our research questions. First, we subset our selected six
plots for our analysis:

\begin{verbatim}
## [1] "DF_3B" "IF_2A" "DF_5A" "DF_5B" "DF_6A" "DF_6B"
\end{verbatim}

These plots were chosen out of the total 20 plots because they were the
only ones that had species identifications attached to samples, which
was needed for our investigation of how defaunation affects species
composition.

Next, we only selected columns that contained variables of interest:

\begin{verbatim}
## [1] "Project.Plot" "Plant_tag"    "DBH_mm"       "Height_m"     "Veg_Type"    
## [6] "ID"
\end{verbatim}

We removed absent or unreasonable values from the dataset. This involved
simply removing blank cells or ``NAs'', as well as measurements that
were likely incorrect, probably as a result of improper unit
conversions. Additionally, we improved uniformity in the dataset by
removing samples that had a height less than 1.5m and lianas. This was
because not all plots measured individuals smaller than 1.5m, and height
measurements for lianas are less reliable, so we decided to only analyze
trees. We also found strange instances in the data: some samples were
relatively tall yet had very low DBHs (Figure \_\_); this is probably
due to some error in units. Therefore, we removed any samples that had a
DBH less than 1mm and a height above 1.5m to improve accuracy.

\begin{figure}
\centering
\includegraphics{GoldenGriffithsKnierMalinowski_ENV872_Project_files/figure-latex/plot of DBH vs height-1.pdf}
\caption{Height vs.~DBH Comparison}
\end{figure}

Figure \_\_\_? shows that there are clearly errors in the data. The
relationship between height and DBH should follow generally a positive
linear trend especially in early life stages for a tree through
seedling, sapling, juvenile, and early adult life stages. As trees
mature they may slow their growth and reach an aysmptote or threshold
for their height, but may continue to increase in diameter slowly. This
trend is not shown for a large portion of the data. Of particular
concern is the large spread of heights at small DBH values. It is
unrealistic to believe that a 40m tree may have a DBH as small as 10mm.
Therefore, it is likely that there is a high level of error within the
dataset. This error may have occurred in 3 places. First, during data
collection in the field the measurement may have been taken incorrectly.
Secondly, the data may have been recorded incorrectly in the field or
lastly when data was transferred from the paper data collection sheets
into excel it may have been entered incorrectly. This is likely an issue
of unit conversion and will ideally be resolved by checking the dataset
against the raw data sheets. However, this peculiar relationship between
height and DBH will impact the results of this analysis and may explain
why many of the results do not fit with what was expected.

We added in variables to support our research questions and analyses. We
created two new columns: ``Status'' and ``Distance\_km''

\begin{longtable}[]{@{}llll@{}}
\caption{Added Variables to Dataset}\tabularnewline
\toprule
\begin{minipage}[b]{0.10\columnwidth}\raggedright
Variable\strut
\end{minipage} & \begin{minipage}[b]{0.51\columnwidth}\raggedright
Description\strut
\end{minipage} & \begin{minipage}[b]{0.10\columnwidth}\raggedright
Unit\strut
\end{minipage} & \begin{minipage}[b]{0.17\columnwidth}\raggedright
Range\strut
\end{minipage}\tabularnewline
\midrule
\endfirsthead
\toprule
\begin{minipage}[b]{0.10\columnwidth}\raggedright
Variable\strut
\end{minipage} & \begin{minipage}[b]{0.51\columnwidth}\raggedright
Description\strut
\end{minipage} & \begin{minipage}[b]{0.10\columnwidth}\raggedright
Unit\strut
\end{minipage} & \begin{minipage}[b]{0.17\columnwidth}\raggedright
Range\strut
\end{minipage}\tabularnewline
\midrule
\endhead
\begin{minipage}[t]{0.10\columnwidth}\raggedright
Status\strut
\end{minipage} & \begin{minipage}[t]{0.51\columnwidth}\raggedright
Indicates whether each plot is defaunated or intact forest\strut
\end{minipage} & \begin{minipage}[t]{0.10\columnwidth}\raggedright
Category\strut
\end{minipage} & \begin{minipage}[t]{0.17\columnwidth}\raggedright
Defaunated - Intact\strut
\end{minipage}\tabularnewline
\begin{minipage}[t]{0.10\columnwidth}\raggedright
Distance\_km\strut
\end{minipage} & \begin{minipage}[t]{0.51\columnwidth}\raggedright
Distance of each plot from Mokokou\strut
\end{minipage} & \begin{minipage}[t]{0.10\columnwidth}\raggedright
Kilometers\strut
\end{minipage} & \begin{minipage}[t]{0.17\columnwidth}\raggedright
8.177 - 40.224\strut
\end{minipage}\tabularnewline
\bottomrule
\end{longtable}

The categorical variable, ``Status'', will help with data visualization,
and the ``Distance\_km'' variable will be used as a proxy from the
defaunation gradient in our analyses.

The new dimensions of this dataset are:

\begin{verbatim}
## [1] 6479    8
\end{verbatim}

Our cleaned dataset looks as follows:

\begin{longtable}[]{@{}llrrlllr@{}}
\caption{Cleaned Dataset}\tabularnewline
\toprule
Project.Plot & Plant\_tag & DBH\_mm & Height\_m & Veg\_Type & ID &
Status & Distance\_km\tabularnewline
\midrule
\endfirsthead
\toprule
Project.Plot & Plant\_tag & DBH\_mm & Height\_m & Veg\_Type & ID &
Status & Distance\_km\tabularnewline
\midrule
\endhead
DF\_3B & 1554 & 558.8 & 30.0 & Tree & Heisteria parvifolia & Defaunated
& 20.195\tabularnewline
DF\_3B & 69 & 15.8 & 2.4 & Tree & Dialium pachyphyllum & Defaunated &
20.195\tabularnewline
DF\_3B & 4371 & 11.7 & 1.8 & Sapling & Scorodophloeus zenkeri &
Defaunated & 20.195\tabularnewline
DF\_3B & 607 & 19.4 & 2.5 & Tree & Odjendja gabonensis & Defaunated &
20.195\tabularnewline
DF\_3B & 7150 & 21.5 & 3.7 & Tree & Scorodophloeus zenkeri & Defaunated
& 20.195\tabularnewline
DF\_3B & 7110 & 65.0 & 7.5 & Tree & Centroplacus glaucinus & Defaunated
& 20.195\tabularnewline
\bottomrule
\end{longtable}

Accidental misspellings are common in datasets such as this with
thousands of manual entries of complex Latin species names. This is a
concern because two samples that are supposed to be the same species,
but have different spellings, will not be identified as the same species
in our analyses. By looking at a list of the unique species names in the
dataset, we found this to be the case in several instances. Identifying
these errors and correcting them was very labor intensive as this can
only really be done with the human eye using personal judgment as to
what names were meant to be the same. Before cleaning the species names,
there were 349 ``species''; after correcting for spelling mistakes,
there were only 323 species. This means that 26 ``species'' were falsely
identified prior to data cleaning.

The dimensions of the processed, clean dataset are as follows:

\begin{verbatim}
## [1] 6340    7
\end{verbatim}

\newpage

\hypertarget{exploratory-analysis}{%
\section{Exploratory Analysis}\label{exploratory-analysis}}

\hypertarget{dbh}{%
\subsection{DBH}\label{dbh}}

\begin{figure}
\centering
\includegraphics{GoldenGriffithsKnierMalinowski_ENV872_Project_files/figure-latex/unnamed-chunk-5-1.pdf}
\caption{DBH per Site}
\end{figure}

\begin{figure}
\centering
\includegraphics{GoldenGriffithsKnierMalinowski_ENV872_Project_files/figure-latex/unnamed-chunk-7-1.pdf}
\caption{DBH Across Sites}
\end{figure}

\textless\textless\textless\textless\textless\textless\textless{} HEAD
\#LMs

So it seems like most sites are pretty not uniform, and small trees are
over-represented at each location. Let's compare sites to see which site
has the most species.

\includegraphics{GoldenGriffithsKnierMalinowski_ENV872_Project_files/figure-latex/unnamed-chunk-9-1.pdf}
So from this we can see that there are no great disparities between
sites when it comes to species. At 146, DF\_6B has the most species and
at 96 DF\_5B has the fewest species. Most sites have right around 100.
We can also see how each site compares in terms of basal area - i.e.,
how densely forested each site is. IF\_2A has the lowest basal area
where DF\_5A has the greatest at around 15 square meters per hectare.
Given IF\_2A's high species richness but low basal area, these data seem
to suggest that IF\_2A has many small trees but few large ones. Let's
continue our exploration by seeing which genuses contribute the most to
each site's basal area.

\begin{figure}
\centering
\includegraphics{GoldenGriffithsKnierMalinowski_ENV872_Project_files/figure-latex/unnamed-chunk-10-1.pdf}
\caption{DF\_6A Tree Plot}
\end{figure}

\begin{figure}
\centering
\includegraphics{GoldenGriffithsKnierMalinowski_ENV872_Project_files/figure-latex/unnamed-chunk-11-1.pdf}
\caption{IF\_2A Tree Plot}
\end{figure}

From these graphs we can see which genera make up the majority of basal
area at each site. So how similar are these sites to one another? I
wonder. ANOVA?

Let's begin to see if there is any relationship between basal area,
species richness, and distance to towns (i.e., along the defaunation
gradient)?.

\begin{figure}
\centering
\includegraphics{GoldenGriffithsKnierMalinowski_ENV872_Project_files/figure-latex/unnamed-chunk-12-1.pdf}
\caption{Basal Area and Distance to Developed Area}
\end{figure}

Looking at these graphs, it doesn't appear that there's much of an
obvious relationship between basal area, species richness, and distance
along the defaunation gradient. Still, looks can be deceiving. Let's
feed these data into a linear model to see what relationships can be
statistically proved. We begin by checking for correlations among
variables with a corrplot.

from this it appears that basal area per hectare is negatively
correlated with mean distance at the same time, there appears to be a
weak positive correlation between total species and mean distance. Let's
build a model and see how well these correlations predict basal area and
species richness.

\hypertarget{analysis}{%
\section{Analysis}\label{analysis}}

In this linear model we use species richness and distance from developed
area to predict the basal area per hectare of a given site. The null
hypothesis is that there is no relationship between basal area per
hectare, species richness, and distance to town. The alternative
hypothesis is that either both species richness and/or distance to town
will influence basal area per hectare. The results of the model (p =
0.16) indicate that no such significant (p\textless0.05) relationships
exist in this subset of data. However, based on these observations (n =
6), there does appear to be a weak negative relationship (p = 0.09)
between distance to town and basal area per hectare. According to the
model, with every additional kilometer of distance there is an
associated 1.3 square meter decline in basal area of forested land. Once
again, these relationships are not considered significant, which means
the explanatory variables included in this model are not sufficient to
explain observed patterns in stand density along the defaunation
gradient. There is no relationship between distance to town and species
richness (p = 0.74).

\hypertarget{question-1-insert-specific-question-here-and-add-additional-subsections-for-additional-questions-below-if-needed}{%
\subsection{Question 1: \textless insert specific question here and add
additional subsections for additional questions below, if
needed\textgreater{}}\label{question-1-insert-specific-question-here-and-add-additional-subsections-for-additional-questions-below-if-needed}}

\hypertarget{question-2}{%
\subsection{Question 2:}\label{question-2}}

\newpage

\hypertarget{summary-and-conclusions}{%
\section{Summary and Conclusions}\label{summary-and-conclusions}}

Our data cleaning and analyses have provided an essential starting point
to further explore the effects of defaunation on tropical forest
structure and composition within Ivindo National Park Gabon. Although
much of the results differ from our expected outcomes a key finding in
this exploration has been uncovering errors in the dataset that may lead
to faulty results or uninterpretable findings. We were able to address
our \emph{overarching research question} - ``How does defaunation affect
forest structure and composition in Ivindo National Park, Gabon?''
although further data collection, checking, and analysis is necessary.

\emph{RQ1 \& RQ2 - Forest Structure} Does average diameter at breast
height (DBH) change along a defaunation gradient?

Are there differences in Basal Area across a defaunation gradient?

\emph{RQ3 - Species Composition} Does species composition change along a
defaunation gradient?

Overall, no specific trends were found between defaunation (using
distance to village as a proxy for defaunation) and our forest structure
and composition variables (DBH, basal area, and genus richness). In
addition to the discrepencies in the DBH and height data another factor
contributing to lack of significant results may be that amongst the 6
plots used in our analyses only one was considered intact forest. A
greater dataset including the remaming plots along the gradient may help
to draw more conclusions about these relationships and uncover any
trends. There is currently a team of Gabonese researchers and colleagues
from the Poulsen Lab conducting this work in Gabon. Re-analyses will be
necessary once a completed dataset with all plots and all plant IDs is
cleaned and available. Just as important as understanding defaunation
implications for carbon sequestration in tropical forests is the need
for good, reliable data to draw conclusions to advance policy and
management.

\newpage

\hypertarget{references}{%
\section{References}\label{references}}

\begin{enumerate}
\def\labelenumi{\arabic{enumi}.}
\item
  Kurten, E. L. Cascading effects of contemporaneous defaunation on
  tropical forest communities. Biol. Conserv. 163, 22--32 (2013).
\item
  Culot, L., Bello, C., Batista, J. L. F., do Couto, H. T. Z. \&
  Galetti, M. Synergistic effects of seed disperser and predator loss on
  recruitment success and long-term consequences for carbon stocks in
  tropical rainforests. Sci. Rep.~7, 7662 (2017).
\item
  Beaune, D. et al.~Seed dispersal strategies and the threat of
  defaunation in a Congo forest. Biodivers. Conserv. 22, 225--238
  (2013).
\item
  Bello, C. et al.~Defaunation affects carbon storage in tropical
  forests. Sci. Adv. 1, e1501105 (2015).
\item
  Koerner, S. E., Poulsen, J. R., Blanchard, E. J., Okouyi, J. \& Clark,
  C. J. Vertebrate community composition and diversity declines along a
  defaunation gradient radiating from rural villages in Gabon. J. Appl.
  Ecol. 54, 805--814 (2017).
\end{enumerate}

\end{document}
